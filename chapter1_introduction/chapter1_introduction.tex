\section{はじめに}\label{chap-1-introduction}
  圏論は代数学でいう代数と準同型写像を扱う代数で、プログラミングで言えば型と関数の理論と呼ぶことができる。
  どちらも既存の理論のモデル化であるが、圏論では「対象」が内部でどのように構成されるかではなく、「対象」が外部に対してどのように振る舞うかを重視する傾向にある。それによって圏論は抽象化能力に優れており、数学にとどまらず様々な分野に応用されている。
  当然抽象的なだけではなく圏論は体系的で単純な理論である。圏論に登場する様々な概念は圏論によって一般化されるし、更に単純な概念によって構成できることが多い。
	本資料では数学基礎論や計算機科学で特に使われ、とりわけ体系的で比較的単純であるカルテジアン閉圏を目標に解説していく。
\subsection{このpdfについて}\label{chap-1.1-about-this-pdf}
他の入門書の差別化として、できるだけ議論や具体例を圏論の中で完結するようにしている。圏論の概念は圏論で一般化、構成できることを述べたが、この入門書ではそれらを目的として議論を進めていく。
それ故に具体例を後回しで紹介することが多いため、どうしても気になるのであれば適宜飛ばして読んでほしい。\\
また証明もほとんど省略せず、行間ができないように記述してある。そのため全体的に煩雑に見えるかもしれないが、個々の操作は単純であるため自明であると感じた場合はいちいち証明を追わなくてもよい。\\